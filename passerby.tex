%-----------------------------------------------------------------------
% Beginning of article-template.tex
%-----------------------------------------------------------------------
%
%    This is a template file for proceedings articles prepared with AMS
%    author packages, for use with AMS-LaTeX.
%
%    Templates for various common text, math and figure elements are
%    given following the \end{document} line.
%
%%%%%%%%%%%%%%%%%%%%%%%%%%%%%%%%%%%%%%%%%%%%%%%%%%%%%%%%%%%%%%%%%%%%%%%%

%    Remove any commented or uncommented macros you do not use.

%    Replace amsproc by the name of the author package.
\documentclass{amsart}

%    If you need symbols beyond the basic set, uncomment this command.
%\usepackage{amssymb}

%    If your article includes graphics, uncomment this command.
%\usepackage{graphicx}

%    If the article includes commutative diagrams, ...
%\usepackage[cmtip,all]{xy}

%    Include other referenced packages here.
% \usepackage{}
\usepackage{hyperref}
\usepackage{algorithm}
\usepackage{algpseudocode}
\usepackage{fancyvrb}
\usepackage{amsmath}
% See http://tex.stackexchange.com/questions/32051/variable-sized-such-that-pipe
% For usage of /Set that depends on braket
\usepackage{braket}
% For showing images:
% \usepackage{svg}
% For showing text images:
% \usepackage{pstricks}

% For showing images:
\usepackage{graphicx}
% \usepackage{svg}

%    Update the information and uncomment if AMS is not the copyright
%    holder.
%\copyrightinfo{2009}{American Mathematical Society}

\newtheorem{theorem}{Theorem}[section]
\newtheorem{lemma}[theorem]{Lemma}

\theoremstyle{definition}
\newtheorem{definition}[theorem]{Definition}
\newtheorem{example}[theorem]{Example}
\newtheorem{xca}[theorem]{Exercise}

\theoremstyle{remark}
\newtheorem{remark}[theorem]{Remark}

\numberwithin{equation}{section}

\DeclareMathOperator*{\argmin}{arg\,min}
\newcommand{\divides}{\mid}

\begin{document}


\title{Passerby messaging system}
%    Only \author and \address are required; other information is
%    optional.  Remove any unused author tags.

%    author one information
% \author[short version for running head]{name for top of paper}
\author{real}
% \address{freedomlayer}
% \curraddr{}
\email[real]{real@freedomlayer.org}
\thanks{Freedomlayer research facility}

%    author two information
% \author{}
% \address{}
% \curraddr{}
% \email{}
% \thanks{}

% \subjclass[2000]{Primary}
%    The 2010 edition of the Mathematics Subject Classification is
%    now available.  If you are citing a classification from the
%    new scheme, use the following input coding instead.
% See http://www.ams.org/msc/msc2010.html?t=05C30&btn=Current for explanation.
\subjclass[2010]{Primary 05C30}

\date{22.06.2017}
\maketitle

\begin{abstract}
  Passerby is an experimental messaging system that allows real time anonymous
  and encrypted messaging.
  This document described the protocol being used in Passerby.

\end{abstract}
\section{Intro}
% \subsection{Subsection}

Consider a connected network of $n$ computers. Some pairs of computers have a
direct connection between them. We want to be able to pass messages
between any two computers on the network using the existing direct connections.

Assuming that a message originates from some computer, at each stage the current
computer passes the message to one of his neighbors. Eventually the message
should arrive at its destination computer. We also assume that every computer
on the network has only limited amount of memory, and hence can not comprehend
the full structure of the network.

We can think of the network structure as a simple graph, where vertices
represent computers in the network. Two vertices in the graph have an edge
between them if the two corresponding computers are directly connected.

In this document we propose a distributed algorithm for routing in connected
networks which is based on Chord's overlay structure \cite{chord-stoica}. Our
design builds upon the work in \cite{pushing-chord-overlay,
virtual-ring-routing}.

% Explain about Chord? What is a DHT?

Our main contribution is a new design to the overlay structure, and an analysis
of the resulting overlay connectivity. We prove that our new design always
results in a single connected overlay ring structure.



% - Read about Moni Naor Know thy neighbor, and find an even faster routing in
%   Chord. Learn how to prove it, and include the proof in this paper.

%    Bibliographies can be prepared with BibTeX using amsplain,
%    amsalpha, or (for "historical" overviews) natbib style.
\bibliographystyle{amsplain}
%    Insert the bibliography data here.
\begin{thebibliography}{9}
\bibitem{chord-stoica}
Ion Stoica, Robert Morris, David Karger, M. Frans Kaashoek, Hari Balakrishnan
\textit{Chord: A scalable Peer-to-peer Lookup Service for Internet Applications}
% \\\texttt{http://www.math.ucsd.edu/\~{}ronspubs/81\_04\_small\_trees.pdf}

\bibitem{pushing-chord-overlay}
Thomas Fuhrmann, Pengfei Di, Kendy Kutzner, Curt Cramer
\textit{Pushing Chord into the Underlay: Scalable Routing for Hybrid MANETs}

\bibitem{virtual-ring-routing}
  Matthew Caesar, Miguel Castro, Edmund B. Nightingale, Greg O'Shea, Antony Rowstron
\textit{Virtual Ring Routing: Network Routing Inspired by DHTs}
\\

\end{thebibliography}

\end{document}

