%-----------------------------------------------------------------------
% Beginning of article-template.tex
%-----------------------------------------------------------------------
%
%    This is a template file for proceedings articles prepared with AMS
%    author packages, for use with AMS-LaTeX.
%
%    Templates for various common text, math and figure elements are
%    given following the \end{document} line.
%
%%%%%%%%%%%%%%%%%%%%%%%%%%%%%%%%%%%%%%%%%%%%%%%%%%%%%%%%%%%%%%%%%%%%%%%%

%    Remove any commented or uncommented macros you do not use.

%    Replace amsproc by the name of the author package.
\documentclass{amsart}

%    If you need symbols beyond the basic set, uncomment this command.
%\usepackage{amssymb}

%    If your article includes graphics, uncomment this command.
%\usepackage{graphicx}

%    If the article includes commutative diagrams, ...
%\usepackage[cmtip,all]{xy}

%    Include other referenced packages here.
% \usepackage{}
\usepackage{hyperref}
\usepackage{algorithm}
\usepackage{algpseudocode}
\usepackage{fancyvrb}
\usepackage{amsmath}
% See http://tex.stackexchange.com/questions/32051/variable-sized-such-that-pipe
% For usage of /Set that depends on braket
\usepackage{braket}
% For showing images:
% \usepackage{svg}
% For showing text images:
% \usepackage{pstricks}

% For showing images:
\usepackage{graphicx}
% \usepackage{svg}

%    Update the information and uncomment if AMS is not the copyright
%    holder.
%\copyrightinfo{2009}{American Mathematical Society}

\newtheorem{theorem}{Theorem}[section]
\newtheorem{lemma}[theorem]{Lemma}

\theoremstyle{definition}
\newtheorem{definition}[theorem]{Definition}
\newtheorem{example}[theorem]{Example}
\newtheorem{xca}[theorem]{Exercise}

\theoremstyle{remark}
\newtheorem{remark}[theorem]{Remark}

\numberwithin{equation}{section}

\DeclareMathOperator*{\argmin}{arg\,min}
\newcommand{\divides}{\mid}

\begin{document}


\title{Passerby messaging system}
%    Only \author and \address are required; other information is
%    optional.  Remove any unused author tags.

%    author one information
% \author[short version for running head]{name for top of paper}
\author{real}
% \address{freedomlayer}
% \curraddr{}
\email[real]{real@freedomlayer.org}
\thanks{Freedomlayer research facility}

%    author two information
% \author{}
% \address{}
% \curraddr{}
% \email{}
% \thanks{}

% \subjclass[2000]{Primary}
%    The 2010 edition of the Mathematics Subject Classification is
%    now available.  If you are citing a classification from the
%    new scheme, use the following input coding instead.
% See http://www.ams.org/msc/msc2010.html?t=05C30&btn=Current for explanation.
\subjclass[2010]{Primary 05C30}

\date{22.06.2017}
\maketitle

\begin{abstract}
  Passerby is an experimental messaging system that allows real time anonymous
  and encrypted messaging. This document describes the protocol being used in
  Passerby.
\end{abstract}

\section{Main goals}

Main goals of the Passerby system:

\begin{enumerate}
  \item Passing messages between people with high probability of success.
  \item Encrypting the messages passed between people.
  \item Keeping a contact list.
\end{enumerate}


\section{Low level API messages}

Possible messages to send to the Passerby router:
\begin{enumerate}
  \item $CreateChannel (remote_id)$
  \item $SendMessage (remote_id, datagram_data)$
  \item $CloseChannel (remote_id)$
\end{enumerate}

Possible messages to receive from the Passerby router:
\begin{enumerate}
  \item $MessageReceived (remote_id, datagram_data)$
\end{enumerate}

Node ids are of size 120 bits (15 bytes). They are a hash of a public key.

\section{Operation of the Passerby router}

\subsection{Messages between routers}

The Passerby router has direct reliable TCP like connections to various nodes,
implemented somehow. For any direct connection with a neighboring node, the
following messages can be transferred:

Initial handshake:
\begin{enumerate}
  \item $LocalId(LocalId)$
      First message sent. Introduces the $LocalId$ to the remote side.
  \item $SendEncKey(Enc_{RemoteId}(SendSymKey))$
      Send the key used to encrypt the next messages.
\end{enumerate}

After the initial handshake, all messages are encrypted and MACed with the
$SendSymKey$, with some padding added. The inner structure of a message is as
follows:

\begin{enumerate}
  \item magic
  \item length
  \item Encrypted message
    \begin{enumerate}
      \item random padding
      \item message
    \end{enumerate}
\end{enumerate}

The following messages can be sent after the handshake, encrypted and MACed:

\begin{enumerate}
  \item $UpdatedLandmarkPath(SignedPathToLandmark, TimestampsPath)$
      Used both for new choice of landmark, or changes in the network layout.
      Right after the handshake, this kind of message should be sent for all
      landmarks.
  \item $TimeTick(AggrRandom)$
      A time tick, used for implementing distributed time.
  \item $NetCoordMessage(RemoteNetCoord, RemoteId, NCMessage)$
    Send a message to a remote node according to RemoteNetCoord.
\end{enumerate}

% TODO: Explain about the structure of NCMessage here.
% Should include how to create and share symmetric encryption key.

The following messages are used to maintain the DHT:
\begin{enumerate}
  \item $DHTFollow(Fingers)$
  \item $ResponseFingers(Fingers)$
\end{enumerate}

% \subsection{Subsection}


%    Bibliographies can be prepared with BibTeX using amsplain,
%    amsalpha, or (for "historical" overviews) natbib style.
\bibliographystyle{amsplain}
%    Insert the bibliography data here.
\begin{thebibliography}{9}
\bibitem{chord-stoica}
Ion Stoica, Robert Morris, David Karger, M. Frans Kaashoek, Hari Balakrishnan
\textit{Chord: A scalable Peer-to-peer Lookup Service for Internet Applications}
% \\\texttt{http://www.math.ucsd.edu/\~{}ronspubs/81\_04\_small\_trees.pdf}

\bibitem{pushing-chord-overlay}
Thomas Fuhrmann, Pengfei Di, Kendy Kutzner, Curt Cramer
\textit{Pushing Chord into the Underlay: Scalable Routing for Hybrid MANETs}

\bibitem{virtual-ring-routing}
  Matthew Caesar, Miguel Castro, Edmund B. Nightingale, Greg O'Shea, Antony Rowstron
\textit{Virtual Ring Routing: Network Routing Inspired by DHTs}
\\

\end{thebibliography}

\end{document}

